\section{Properties}
\label{property}

 \begin{lemma}
     $\mathrm{(Concrete Function)}$. if $\upsilon$ is a value of type $\tau_1\rightarrow\tau_2$, then $\upsilon= \lambda x:\tau_1.e$, where e has type $\tau_2$.
    \end{lemma}
 \begin{proof}
     According to the $\lambdaJ$ syntax, we can get Lemma 1 immediately.
    \end{proof}
 
 \begin{theorem}
     $\mathrm{(Progress).}$ Suppose $e$ is a closed, well-typed expression. Then $e$ is either a value $\upsilon$ or there is some $e'$ such that $\vdash\left\langle\phi ,\phi ,e\right\rangle\rightarrow\left\langle\Sigma',\Delta',e'\right\rangle$.
 \end{theorem}   
    \begin{proof}
        we can ...
    \end{proof}
    
    \begin{theorem}
        $\mathrm{(Preservision).}$ If $\Gamma\vdash e:\tau\delta$ and $e\rightarrow e'$, then $\Gamma\vdash e':\tau\delta$. 
    \end{theorem}
    \begin{proof}
        we can ...
    \end{proof}