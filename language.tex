\section{Language Design}
\label{language}
 \subsection{Jeeves}
 Here we discribe Jeeves syntax in Figure \ref{jeeve-syntax}. Totally, it contains three types 
   \begin{figure}[!htbp]
   \fbox{%  
       \parbox{\textwidth}{%  
           \begin{center}  
                \begin{align*}
                   Level~:: = ~ & {\begin{aligned}[t]
                       & ~ \bot~|~\top\\                      
                       \end{aligned}}\\
                  Exp~:: = ~ & {\begin{aligned}[t]
                           & ~ \upsilon~|~Exp_1~(op)~Exp_2
                           \\
                      ~|~& ~ \textbf{if}~ Exp_1~\textbf{then}~Exp_t~\textbf{else}~Exp_f
                      \\ 
                      ~|~& ~ Exp_1~Exp_2
                      \\
                      ~|~& ~ \left\langle Exp_{\bot}~|~Exp_{\top}\right\rangle(\ell)
                      \\
                      ~|~& ~ \textbf{level}~\ell~\textbf{in}~Exp
                      \\
                      ~|~& ~ \textbf{policy}~\ell:~Exp_p~\textbf{then}~Level~\textbf{in}~Exp
                      \end{aligned}}\\
                Stmt~:: = ~ & {\begin{aligned}[t]
                    & ~ \textbf{let}~x:\tau~=~Exp
                    \\
                    ~|~& ~ \textbf{print}~\{Exp_c\}~Exp
                    \end{aligned}}
                \end{align*}
            \end{center}  
        }%  
    }
    \caption{Jeeves syntax}
    \label{jeeve-syntax}
\end{figure}
\subsection{Lambda J}
    Here we discribe the $\lambdaJ$ language show in Figure \ref{lambdaj-syntax}.\\
   \begin{figure}[!htbp]
   \fbox{%  
       \parbox{\textwidth}{%  
           \begin{center}  
               \begin{align*}
                   c~:: = ~ & {\begin{aligned}[t]
                                & ~ n~|~b~|~\lambda x:\tau .e~|~record~x\bar{:}\upsilon
                                \\
                           ~|~& ~ \textbf{error}~|~()
                               \end{aligned}}\\
           \sigma~:: = ~ & {\begin{aligned}[t]
                                & ~ x~|~\textbf{contex}~\tau
                                \\
                          ~|~& ~ c_1~(op)~\sigma_2~|~\sigma_1~(op)~c_2
                          \\ 
                          ~|~& ~ \sigma_1~(op)~\sigma_2
                          \\
                          ~|~& ~ \textbf{if}~\sigma~\textbf{then}~\upsilon_t~\textbf{else}~\upsilon_f
                           \end{aligned}}\\
           \upsilon~:: = ~ & {\begin{aligned}[t]
                               & ~ c~|~\sigma
                               \end{aligned}}\\
                      e~:: = ~ & {\begin{aligned}[t]
                               & ~ \upsilon~|~e_1~(op)~e_2
                            \\
                            ~|~& ~ \textbf{if}~e_1~\textbf{then}~e_t~\textbf{else}~e_f~|~e_1~e_2
                            \\
                            ~|~& ~ \textbf{let}~x:\tau = e_1~\textbf{in}~e_2
                            \\
                            ~|~& ~ \textbf{let rec}~f:\tau = e_1~\textbf{in}~e_2
                            \\
                            ~|~& ~ \textbf{defer}~x:\tau\{e\}~\textbf{defaut}~\upsilon_d
                            \\
                            ~|~& ~ \textbf{assert}~e
                            \\
                            ~|~& ~ \textbf{concretize}~e~\textbf{with}~\upsilon_c
                    \end{aligned}}
                \end{align*}
            \end{center}  
        }%  
    }
    \caption{$\lambdaJ$ syntax}
    \label{lambdaj-syntax}
    \end{figure}
   