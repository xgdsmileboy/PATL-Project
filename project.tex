\documentclass[a4paper,11pt]{article}
%\usepackage{CJKutf8}
%\usepackage[utf8]{inputenc}
\usepackage[top=1in, bottom=1in, left=1.25in, right=1.25in]{geometry}

\usepackage{amsmath}
\usepackage{amsthm}
\usepackage{amssymb}
\usepackage{xspace}
\usepackage{listings}
\usepackage{xcolor}
\usepackage{graphicx} 
\usepackage{stfloats}
\usepackage{bussproofs}
\usepackage{amssymb}
\usepackage{latexsym}
\usepackage{fancyhdr}
\usepackage{hyperref}
\usepackage{graphics}
\usepackage{stackengine}
\usepackage[english]{babel}
\usepackage{amsmath,amsthm}
\usepackage{mathrsfs}
\usepackage{setspace}

\newcommand\lambdaJ{{\ensuremath{\lambda_{\text{J}}}}}
\newtheorem{lemma}{Lemma}
\newtheorem{theorem}{Theorem}
\newtheorem{definition}{Definition}
\newtheorem{property}{Property}


\lstset{%numbers=left,
    language=Haskell,
    numberstyle=\small,
    keywordstyle=\color{blue},
    commentstyle=\color[cmyk]{1,0,1,0},
    basicstyle=\ttfamily,
    escapeinside=@@,
    breaklines,
    extendedchars=false,
    tabsize=4,
    showspaces=false
}

%\begin{premaker} 
%    % The premaker environment contains Title, authors and addresses;
%    % use the thanksref command within \title, \author or \address for footnotes;
%    % use the corauthref command within \author for corresponding author footnotes;
%    % use the ead command for the email address,
%    % and the form \ead[url] for the home page: 
%    
%    \title{Secure and Efficient Information Acquisition Mechanism of Internet of Things} 
%    \author[author1]{L P\corauthref{cor1}},
%    \ead{lp@163.com}
%    \corauth[cor1]{Corresponding author.}
%    % other authors
%    \author[author2]{P H}
%    \author[author3]{D H}
%    \address[author1]{School of A, B University, C 000000, China}
%    \address[author2]{Address for 2nd author}
%\end{premaker} 

\begin{document}
    \title{Types and Programming Language Project Report\footnote{done at June 5th, 2016}}
    \author{Jiajun Jiang\\ jiajun.jiang@pku.edu.cn\\ 
        \and 
        Mengwei Xu\\ 
        xumw@pku.edu.cn\\ 
    } 
    \date{}
    \maketitle


    
\section*{Introduction}
\label{intro}

In current Internet environment, users tend to share more personal data online. This phenomenon makes it increasingly important for applications to protect confidentiality. For existing approaches to achieve privacy control, programmers are forced to ensure compliance by their own efforts, even when both the application and the policies may be evolving rapidly. This can cause considerable burdens to application developers.

This academic paper, called \textit{A Language for Automatically Enforcing Privacy Policies.}\footnote{http://www.cs.cmu.edu/~jyang2/papers/popl088-yang.pdf}, proposes a new programming model that makes the system responsible for automatically producing outputs consistent with programmer-specified policies. This automation makes it easier for programmers to enforce policies specifying how each sensitive value should be displayed in a given context, therefore solves the problem mentioned above. Furthermore, they have implemented this programming model in a new functional constraint language named \textbf{Jeeves}.

We carried out our course project based on this paper. More specifically, we first read and comprehended this paper, including the language design, the semantics, the evaluation and typing rules, the partial property proof and the implementation details as we did in our class. Then we proved the progress and preservation parts which is left out in the paper by ourselves. Finally, we successfully run the implementation codes provided by the authors and wrote our own use cases to understand the implementation details.

\section{Language Design}
\label{language}
 \subsection{Jeeves}
 Here we discribe Jeeves syntax in Figure \ref{jeeve-syntax}. Totally, it contains three types 
   \begin{figure}[!htbp]
   \fbox{%  
       \parbox{\textwidth}{%  
           \begin{center}  
                \begin{align*}
                   Level~:: = ~ & {\begin{aligned}[t]
                       & ~ \bot~|~\top\\                      
                       \end{aligned}}\\
                  Exp~:: = ~ & {\begin{aligned}[t]
                           & ~ \upsilon~|~Exp_1~(op)~Exp_2
                           \\
                      ~|~& ~ \textbf{if}~ Exp_1~\textbf{then}~Exp_t~\textbf{else}~Exp_f
                      \\ 
                      ~|~& ~ Exp_1~Exp_2
                      \\
                      ~|~& ~ \left\langle Exp_{\bot}~|~Exp_{\top}\right\rangle(\ell)
                      \\
                      ~|~& ~ \textbf{level}~\ell~\textbf{in}~Exp
                      \\
                      ~|~& ~ \textbf{policy}~\ell:~Exp_p~\textbf{then}~Level~\textbf{in}~Exp
                      \end{aligned}}\\
                Stmt~:: = ~ & {\begin{aligned}[t]
                    & ~ \textbf{let}~x:\tau~=~Exp
                    \\
                    ~|~& ~ \textbf{print}~\{Exp_c\}~Exp
                    \end{aligned}}
                \end{align*}
            \end{center}  
        }%  
    }
    \caption{Jeeves syntax}
    \label{jeeve-syntax}
\end{figure}
\subsection{Lambda J}
    Here we discribe the $\lambdaJ$ language show in Figure \ref{lambdaj-syntax}.\\
   \begin{figure}[!htbp]
   \fbox{%  
       \parbox{\textwidth}{%  
           \begin{center}  
               \begin{align*}
                   c~:: = ~ & {\begin{aligned}[t]
                                & ~ n~|~b~|~\lambda x:\tau .e~|~record~x\bar{:}\upsilon
                                \\
                           ~|~& ~ \textbf{error}~|~()
                               \end{aligned}}\\
           \sigma~:: = ~ & {\begin{aligned}[t]
                                & ~ x~|~\textbf{contex}~\tau
                                \\
                          ~|~& ~ c_1~(op)~\sigma_2~|~\sigma_1~(op)~c_2
                          \\ 
                          ~|~& ~ \sigma_1~(op)~\sigma_2
                          \\
                          ~|~& ~ \textbf{if}~\sigma~\textbf{then}~\upsilon_t~\textbf{else}~\upsilon_f
                           \end{aligned}}\\
           \upsilon~:: = ~ & {\begin{aligned}[t]
                               & ~ c~|~\sigma
                               \end{aligned}}\\
                      e~:: = ~ & {\begin{aligned}[t]
                               & ~ \upsilon~|~e_1~(op)~e_2
                            \\
                            ~|~& ~ \textbf{if}~e_1~\textbf{then}~e_t~\textbf{else}~e_f~|~e_1~e_2
                            \\
                            ~|~& ~ \textbf{let}~x:\tau = e_1~\textbf{in}~e_2
                            \\
                            ~|~& ~ \textbf{let rec}~f:\tau = e_1~\textbf{in}~e_2
                            \\
                            ~|~& ~ \textbf{defer}~x:\tau\{e\}~\textbf{defaut}~\upsilon_d
                            \\
                            ~|~& ~ \textbf{assert}~e
                            \\
                            ~|~& ~ \textbf{concretize}~e~\textbf{with}~\upsilon_c
                    \end{aligned}}
                \end{align*}
            \end{center}  
        }%  
    }
    \caption{$\lambdaJ$ syntax}
    \label{lambdaj-syntax}
    \end{figure}
   

\section{Semantics}
\label{semantics}

The semantics are divided into two parts: dynamic and static. Dynamic semantic is the same as the evaluation rules in our class, while static semantic includes but not limited to the typing rules. The static semantic also includes some rules to rule out the infinite recursive calling that may be caused by the logical variables introduced by $\lambdaJ$.

As shown in Figure \ref{dynamic-sem-lambdaj}, the $\lambdaJ$ evaluation rules extend {\ensuremath{\lambda}}-calculus evaluation with constraint propagation and symbolic evaluation of logic variables. Evaluation involves keeping track of constraints which are required to be true (hard constraints) and the set of constraints we use for guidance if consistent with our hard constraints (default assumptions). To correctly evaluate conditionals with symbolic conditions, we also need to keep track of the (possibly symbolic) path condition ($\mathscr{G}$). Here, $\Sigma~$stores the current set of constraints, and an environment $\Delta~$stores the set of constraints on default values for logic variables. The rules are no surprising, quite similar to the rules learned in our class. One interesting thing is when evaluating under symbolic conditions, because evaluation of such conditionals with a recursive function application in a branch could lead to infinite recursion when the condition is symbolic. $\lambdaJ$ prevents this anomalous behavior by using the type system.

Static semantic, as shown in Figure \ref{static-sem-lambdaj}, describes simple type-checking and enforce restriction on scope of nondeterminism and recursion. The types are defined in Figure \ref{lambdaj-types}. The largest difference of typing rules between the normal {\ensuremath{\lambda}}-calculus and $\lambdaJ$ is the symbolic values. For int, bool or if expression, it can have two options: concrete or symbolic. And therefore the issue comes when applying reentrant functions under symbolic conditionals. To rule out such kind of cases, $\lambdaJ$ explicitly marks values as concrete or symbolic. For example, a function that might be reentrant should be marked as ($\longrightarrow$). Furthermore, it introduces \textbf{rep}, defined as a predicate that requires that functions taking arguments that may be reentrant be themselves labeled as reentrant. This can help prevent high-order functions from being used to circumvent the restrictions on reentrant calls. Overall, the static semantics can guarantee that
\begin{itemize}
\item concrete values are supplied when concrete values are expected,
\item symbolic values are well-formed,
\item evaluation under symbolic conditions does not cause unexpected infinite recursion,
\item context values have the appropriate types.
\end{itemize}


\begin{figure}
     \begin{spacing}{2.0}
    \fbox{
        \parbox{\textwidth}{
            \begin{center}
                \scriptsize{
                  $ \boxed{\mathscr{G} \vdash \left \langle \Sigma, \Delta, e \right \rangle \rightarrow \left \langle \Sigma', \Delta', e' \right \rangle }$\\}
            \end{center}
            \begin{flushright}
%                \large{
                  $ \frac{
                      \mathscr{G} \vdash \left \langle \Sigma, \Delta, e_1 \right \rangle \rightarrow \left \langle \Sigma', \Delta', e'_1 \right \rangle 
                      }{
                      \mathscr{G} \vdash \left \langle \Sigma, \Delta, e_1~e_2 \right \rangle \rightarrow \left \langle \Sigma', \Delta', e'_1~e_2 \right \rangle 
                      }~~~~\scriptsize{\textrm{E-APP1}}
                      ~~~~~
                      \frac{
                          \mathscr{G} \vdash \left \langle \Sigma, \Delta, e_2 \right \rangle \rightarrow \left \langle \Sigma', \Delta', e'_2 \right \rangle 
                        }{
                        \mathscr{G} \vdash \left \langle \Sigma, \Delta, \upsilon~e_2 \right \rangle \rightarrow \left \langle \Sigma', \Delta', \upsilon~e_2 \right \rangle 
                    }~~~~\scriptsize{\textrm{E-APP2}}$
                    \\
                    $\frac{
                    }{
                    \mathscr{G} \vdash \left \langle \Sigma, \Delta, \lambda x.e~\upsilon \right \rangle \rightarrow \left \langle \Sigma', \Delta', e[x\mapsto \upsilon] \right \rangle 
                    }~~~~\scriptsize{\textrm{E-APPLAMBDA}}
                    ~~~~
                    \frac{
                        c'=c_1~(op)~c_2
                    }{
                    \mathscr{G} \vdash \left \langle \Sigma, \Delta, c_1~(op)~c_2 \right \rangle \rightarrow \left \langle \Sigma', \Delta', c' \right \rangle 
                    }~~~~\scriptsize{\textrm{E-OP}}$
                    \\
                    $\frac{
                        \mathscr{G} \vdash \left \langle \Sigma, \Delta, e_1 \right \rangle \rightarrow \left \langle \Sigma', \Delta', e'_1 \right \rangle 
                    }{
                    \mathscr{G} \vdash \left \langle \Sigma, \Delta, e_1~(op)~e_2 \right \rangle \rightarrow \left \langle \Sigma', \Delta', e'_1~(op)~e_2 \right \rangle 
                    }~~~~\scriptsize{\textrm{E-OP1}}
                    ~~~~
                    \frac{
                        \mathscr{G} \vdash \left \langle \Sigma, \Delta, e_2 \right \rangle \rightarrow \left \langle \Sigma', \Delta', e'_2 \right \rangle 
                    }{
                    \mathscr{G} \vdash \left \langle \Sigma, \Delta, \upsilon~(op)~e_2 \right \rangle \rightarrow \left \langle \Sigma', \Delta', \upsilon~(op)~e'_2 \right \rangle 
                    }~~~~\scriptsize{\textrm{E-OP2}}$
                    \\
                    $\frac{
                        \mathscr{G} \vdash \left \langle \Sigma, \Delta, e_c \right \rangle \rightarrow \left \langle \Sigma', \Delta', e'_c \right \rangle 
                    }{
                    \mathscr{G} \vdash \left \langle \Sigma, \Delta, \textbf{if}~e_c~\textbf{then}~e_t~\textbf{else}~e_f \right \rangle \rightarrow \left \langle \Sigma', \Delta', \textbf{if}~e'_c~\textbf{then}~e_t~\textbf{else}~e_f \right \rangle 
                    }~~~~\scriptsize{\textrm{E-COND}}$
                    \\
                    $\frac{
                        \mathscr{G} \vdash \left \langle \Sigma, \Delta, e_t \right \rangle \rightarrow \left \langle \Sigma', \Delta', e'_t \right \rangle 
                    }{
                        \mathscr{G} \vdash \left \langle \Sigma, \Delta, \textbf{if true}~ \textbf{then}~e_t~\textbf{else}~e_f \right \rangle \rightarrow \left \langle \Sigma', \Delta', e'_t \right \rangle 
                    }~~~~\scriptsize{\textrm{E-CONDTRUE}}$
                    \\
                    $\frac{
                        \mathscr{G} \vdash \left \langle \Sigma, \Delta, e_f \right \rangle \rightarrow \left \langle \Sigma', \Delta', e'_f \right \rangle 
                    }{
                        \mathscr{G} \vdash \left \langle \Sigma, \Delta, \textbf{if false}~\textbf{then}~e_t~\textbf{else}~e_f \right \rangle \rightarrow \left \langle \Sigma', \Delta', e'_f \right \rangle 
                    }~~~~\scriptsize{\textrm{E-CONDFALSE}}$
                    \\
                    $\frac{
                        \mathscr{G} \vdash \left \langle \Sigma, \Delta, e_t \right \rangle \rightarrow \left \langle \Sigma', \Delta', e'_t \right \rangle 
                    }{
                        \mathscr{G} \vdash \left \langle \Sigma, \Delta, \textbf{if}~\sigma~\textbf{then}~e_t~\textbf{else}~e_f \right \rangle \rightarrow \left \langle \Sigma', \Delta', \textbf{if}~\sigma~\textbf{then}~e'_t~\textbf{else}~e_f \right \rangle 
                    }~~~~\scriptsize{\textrm{E-CONDSYMT}}$
                    \\
                    $\frac{
                        \mathscr{G} \vdash \left \langle \Sigma, \Delta, e_f \right \rangle \rightarrow \left \langle \Sigma', \Delta', e'_f \right \rangle 
                    }{
                        \mathscr{G} \vdash \left \langle \Sigma, \Delta, \textbf{if}~\sigma~\textbf{then}~\upsilon_t~\textbf{else}~e_f \right \rangle \rightarrow \left \langle \Sigma', \Delta', \textbf{if}~\sigma~\textbf{then}~\upsilon_t~\textbf{else}~e'_f \right \rangle 
                    }~~~~\scriptsize{\textrm{E-CONDSYMF}}$          
                    \\
                    $\frac{
                        \mathscr{G} \vdash \left \langle \Sigma, \Delta, e \right \rangle \rightarrow \left \langle \Sigma', \Delta', e' \right \rangle 
                      }{
                          \mathscr{G} \vdash \left \langle \Sigma, \Delta, \textbf{defer}~x:\tau\{e\}~\textbf{default}~\upsilon_d \right \rangle \rightarrow \left \langle \Sigma', \Delta', \textbf{defer}~x:\tau\{e'\}~\textbf{default}~\upsilon_d \right \rangle 
                      }~~~~\scriptsize{\textrm{E-DEFERCONSTRAINT}}$       
                    \\
                    $\frac{
                        fresh~x'
                     }{
                         \mathscr{G} \vdash \left \langle \Sigma, \Delta, \textbf{defer}~x:\tau\{e\}~\textbf{default}~\upsilon_d \right \rangle \rightarrow \left \langle \Sigma\cup\{\mathscr{G}\Rightarrow\upsilon_c[x\mapsto x']\}, \Delta\cup\{\mathscr{G}\Rightarrow x'=\upsilon_d \}, x' \right \rangle 
                     }~~~~\scriptsize{\textrm{E-DEFER}}$       
                    \\
                    $\frac{
                        \mathscr{G} \vdash \left \langle \Sigma, \Delta, e \right \rangle \rightarrow \left \langle \Sigma', \Delta', e' \right \rangle 
                    }{
                        \mathscr{G} \vdash \left \langle \Sigma, \Delta, \textbf{assert}~e \right \rangle \rightarrow \left \langle \Sigma', \Delta', \textbf{assert}~e' \right \rangle 
                    }~~~~\scriptsize{\textrm{E-ASSERTCONSTRAINT}} $
                    \\
                    $\frac{ 
                    }{
                        \mathscr{G} \vdash \left \langle \Sigma, \Delta, \textbf{assert}~\upsilon \right \rangle \rightarrow \left \langle \Sigma\cup\{\mathscr{G}\Rightarrow\upsilon\}, \Delta, () \right \rangle 
                    }~~~~\scriptsize{\textrm{E-ASSERT}}$   
                    \\
                    $\frac{
                        \mathscr{G} \vdash \left \langle \Sigma, \Delta, e \right \rangle \rightarrow \left \langle \Sigma', \Delta', e' \right \rangle 
                    }{
                        \mathscr{G} \vdash \left \langle \Sigma, \Delta, \textbf{concretize}~e~\textbf{with}~\upsilon_c \right \rangle \rightarrow \left \langle \Sigma', \Delta', \textbf{concretize}~e'~\textbf{with}~\upsilon_c \right \rangle 
                    }~~~~\scriptsize{\textrm{E-CONCRETIZEEXP}}$   
                    \\
                    $\frac{
                        \textrm{MODLE}( \Delta, \Sigma\cup\{\mathscr{G}\cap\textbf{context}=\upsilon_c \})=\mathscr{M}~~~c=\mathscr{M}[[\upsilon_\nu]]
                    }{
                        \mathscr{G} \vdash \left \langle \Sigma, \Delta, \textbf{concretize}~\upsilon_\nu~\textbf{with}~\upsilon_c \right \rangle \rightarrow \left \langle \Sigma, \Delta, c \right \rangle 
                    }~~~~\scriptsize{\textrm{E-CONCRETIZESAT}}$   
                    \\
                    $\frac{
                        \textrm{MODLE}( \Delta, \Sigma\cup\{\mathscr{G}\cap\textbf{context}=\upsilon_c \})=\textrm{UNSAT}
                    }{
                        \mathscr{G} \vdash \left \langle \Sigma, \Delta, \textbf{concretize}~\upsilon_\nu~\textbf{with}~\upsilon_c \right \rangle \rightarrow \left \langle \Sigma, \Delta, \textbf{error} \right \rangle 
                    }~~~~\scriptsize{\textrm{E-CONCRETIZEUNSAT}}$   
%                }
           \end{flushright}
        }
    }
\end{spacing}
    \caption{Dynamic semantics for $\lambdaJ$.}
    \label{dynamic-sem-lambdaj}
\end{figure}

   \begin{figure}[!htbp]
       \fbox{%  
           \parbox{\textwidth}{%  
               \begin{center}  
                   \begin{align*}
                   \delta~:: = ~ & {\begin{aligned}[t]
                       & ~ \textbf{concretize}~|~\textbf{sym}
                       ~~~~~~~~
                       ~~~~~~~~~~~~~~~\textrm{determinism tag}                  
                       \end{aligned}}\\
                    \beta~:: = ~ & {\begin{aligned}[t]
                        & ~ \textbf{int}_c~|~\textbf{bool}_c~|~\textbf{unit}~| ~ \textbf{int}~|~\textbf{bool}
                        ~~~~~~~~~~~~~~~\textrm{base type}
                        \end{aligned}}\\
                    \tau~:: = ~ & {\begin{aligned}[t]
                        & ~ \beta~|~\tau_1 \stackrel{nr}{\longrightarrow} \tau_2~|~\tau_1\rightarrow\tau_2
                        ~~~~~~~~~~~~~~~~~~~
                        ~~~~~~~~~~~~~~~\textrm{type}
                        \end{aligned}}
                    \end{align*}
                \end{center}  
            }%  
        }
        \caption{$\lambdaJ$ types}
        \label{lambdaj-types}
    \end{figure}

\begin{figure}
    \begin{spacing}{2.0}
        \fbox{
            \parbox{\textwidth}{
                \begin{center}
                    \scriptsize{
                        $ \boxed{\tau_1<:\tau_2}$}
                \end{center}
                \begin{flushright}
                    $ \frac{
                    }{
                        \tau<:\tau
                    }~~~~\scriptsize{\textrm{S-REFLEXIVE}}
                    ~~~~~
                    \frac{
                    }{
                        \textbf{int}_c<:\textbf{int}
                    }~~~~\scriptsize{\textrm{S-INT}}
                    ~~~~~
                    \frac{
                    }{
                    \textbf{bool}_c<:\textbf{bool}
                    }~~~~\scriptsize{\textrm{S-BOOL}}$
                    \\
                    $\frac{
                    }{
                    \tau_1 \stackrel{nr}{\longrightarrow} \tau_2<:\tau_1\rightarrow\tau_2
                    }~~~~\scriptsize{\textrm{S-RECFUN}}
                    ~~~~~
                    \frac{
                        \tau'_1<:\tau_1~~\tau_2<:\tau'_2
                    }{
                    \tau_1 \rightarrow \tau_2 <: \tau'_1 \rightarrow \tau'_2
                    }~~~~\scriptsize{\textrm{S-FUN}} $
                    \\
                \end{flushright}
                \begin{center}
                    \scriptsize{
                        $ \boxed{\textbf{rep}~\tau}$}
                \end{center}
                \begin{flushright}
                    $ \frac{
                        \textbf{rep}~\tau~~~\tau'<:\tau
                    }{
                        \textbf{rep}~\tau
                    }~~~~\scriptsize{\textrm{OK-SUBTYPE}}
                    ~~~~~
                    \frac{
                    }{
                        \textbf{rep}~\beta
                    }~~~~\scriptsize{\textrm{OK-BASETYPE}}$
                    \\
                    $\frac{
                        \textbf{rep}~\tau_2
                    }{
                    \textbf{rep}~\beta_1 \rightarrow \tau_2
                    }~~~~\scriptsize{\textrm{OK-BASEFUNCTION}}
                    ~~~~~
                    \frac{
                        \textbf{rep}~(\tau_1 \stackrel{nr}{\rightarrow}\tau')~~\textbf{rep}~\tau_2
                    }{
                        \textbf{rep}~(\tau_1 \stackrel{nr}{\rightarrow}\tau') \stackrel{nr}{\rightarrow} \tau_2
                    }~~~~\scriptsize{\textrm{OK-HOFUNCTION}}$
                    \\
                    $\frac{
                        \textbf{rep}~\tau_1\rightarrow \tau_2
                    }{
                        \textbf{rep}~(\tau_1\rightarrow \tau_2) \rightarrow\beta
                    }~~~~\scriptsize{\textrm{OK-RECFUNCTIONBASE}}
                    ~~~~~
                    \frac{
                        \textbf{rep}~\tau_1 \rightarrow \tau_2~~\textbf{rep}~\tau'_1 \rightarrow \tau'_2
                    }{
                        \textbf{rep}~(\tau_1 \rightarrow \tau_2) \rightarrow (\tau'_1 \rightarrow \tau'_2)
                    }~~~~\scriptsize{\textrm{OK-RECFUNCTION}}$
                    \\
                \end{flushright}
                \begin{center}
                    \scriptsize{
                        $ \boxed{\Gamma;\gamma\vdash e:\left\langle \tau, \delta \right\rangle }$}
                \end{center}
                \begin{flushright}
                    $ \frac{
                        x\in \Gamma
                    }{
                        \Gamma;\gamma\vdash x:\Gamma(x)
                    }~~~~\scriptsize{\textrm{T-VAR}}
                    ~~~~~
                    \frac{
                    }{
                        \Gamma;\gamma\vdash n:\textbf{int}_c
                    }~~~~\scriptsize{\textrm{T-INT}}
                    ~~~~~
                    \frac{
                    }{
                    \Gamma;\gamma\vdash b:\textbf{bool}_c
                    }~~~~\scriptsize{\textrm{T-BOOL}}
                    ~~~~~
                    \frac{
                    }{
                    \Gamma;\gamma\vdash ():\textbf{unit}
                    }~~~~\scriptsize{\textrm{T-UNIT}}$
                    \\
                    $\frac{
                        \textbf{rep}~\tau
                    }{
                    \Gamma;\gamma\vdash \textbf{context}~\tau :\tau
                    }~~~~\scriptsize{\textrm{T-CONTEXT}}
                    ~~~~~
                    \frac{
                        \Gamma;\gamma\vdash e_1 :\tau_1~~
                        \Gamma;\gamma\vdash e_2 :\tau_2~~
                        \tau_1,\tau_2 <:\tau~~
                        \textbf{rep}~\tau
                    }{
                        \Gamma;\gamma\vdash e_1~(op)~e_2 :\tau
                    }~~~~\scriptsize{\textrm{T-OP}}$
                    \\
                    $ \frac{
                        \Gamma;\gamma\vdash e :\textbf{bool}_c~~
                        \Gamma;\gamma\vdash e_t :\tau_1~~
                        \Gamma;\gamma\vdash e_f :\tau_2~~
                        \tau_1,\tau_2 <:\tau~~
                        \textbf{rep}~\tau
                    }{
                    \Gamma;\gamma\vdash \textbf{if}~e~\textbf{then}~e_t~\textbf{else}~e_f :\tau
                }~~~~\scriptsize{\textrm{T-CONDC}}$
                \\
                $ \frac{
                     \Gamma;\gamma\vdash e :\textbf{bool}~~
                     \Gamma;\textbf{sym} \vdash e_t :\beta_1~~
                     \Gamma;\textbf{sym} \vdash e_f :\beta_2~~
                      \beta_1,\beta_2 <:\beta_c~~
                     \textbf{rep}~\beta_c
                    }{
                    \Gamma;\gamma\vdash \textbf{if}~e~\textbf{then}~e_t~\textbf{else}~e_f :\beta_c
                }~~~~\scriptsize{\textrm{T-CONDSYM}}$
                \\
                $ \frac{
                    \Gamma,x:\tau_d;\gamma\vdash e :\tau'~~
                    \textbf{rep}~\tau_d~~
                    \textbf{rep}~\tau'
                }{
                    \Gamma;\gamma\vdash (\lambda x:\tau_d.e):\tau_d\rightarrow \tau'
                }~~~~\scriptsize{\textrm{T-LAMBDA}}
                ~~~~~
                 \frac{
                     \Gamma;\gamma\vdash e_1:\tau_1\stackrel{nr}{\rightarrow}\tau_2~~
                     \Gamma,x:\tau_d;\gamma\vdash e_2 :\tau'_1~~
                     \tau'_1<:\tau_1~~
                     \textbf{rep}~\tau_1~~
                     \textbf{rep}~\tau_2
                    }{
                    \Gamma;\gamma\vdash (e_1~e_2):\tau_2
                }~~~~\scriptsize{\textrm{T-APP}}$
                \\
                $\frac{
                    \Gamma,f:\tau_1 \rightarrow \tau_2 ;\gamma\vdash e_1:\tau_1 \rightarrow \tau_2~~
                    \Gamma,f:\tau_1 \stackrel{nr}{\rightarrow} \tau_2 ;\gamma\vdash e_2:\tau_2~~
                    \textbf{rep}~\tau~~
                    \textbf{rep}~\tau_2
                }{
                    \Gamma;\gamma\vdash \textbf{let rec}~f:\tau_1 \stackrel{nr}{\rightarrow} \tau_2=e_1~\textbf{in}~e_2:\tau_2
                }~~~~\scriptsize{\textrm{T-APPLETREC}}$
                \\
                $\frac{
                    \gamma=\textbf{concrete}~~
                    \Gamma;\gamma\vdash e_1:\tau_1 \rightarrow \tau_2~~
                    \Gamma;\gamma\vdash e_2:\tau'_1~~
                    \tau'_1<:\tau_1~~
                    \textbf{rep}~\tau_1~~
                    \textbf{rep}~\tau_2
                }{
                    \Gamma;\gamma\vdash (e_1~e_2):\tau_2
                }~~~~\scriptsize{\textrm{T-APPCURREC}}$
                 \\
                 $ \frac{
                     \Gamma,x:\beta;\gamma\vdash e_c :\textbf{bool}~~
                     \Gamma;\gamma\vdash \upsilon :\beta
                    }{
                        \Gamma;\gamma\vdash (\textbf{defer} x:\beta\{e_c\}~\textbf{default}~\upsilon):\beta
                    }~~~~\scriptsize{\textrm{T-DEFER}}
                    ~~~~~
                    \frac{
                        \Gamma;\gamma\vdash e_c:\textbf{bool}~~
                    }{
                        \Gamma;\gamma\vdash (\textbf{assert}~e_c):\textbf{unit}
                    }~~~~\scriptsize{\textrm{T-ASSERT}}$
                    \\
                    $\frac{
                        \Gamma;\gamma\vdash e_1:\beta~~
                        \Gamma;\gamma\vdash^{c} e_1:\beta'~~
                        \Gamma;\gamma\vdash \upsilon:\beta'
                    }{
                    \Gamma;\gamma\vdash (\textbf{concretize}~e_1~\textbf{with}~\upsilon):\beta_c
                }~~~~\scriptsize{\textrm{T-CONCRETIZE}}$
                \end{flushright}
               }
           }
\end{spacing}
\caption{Static semantics for $\lambdaJ$ describing simple type-checking and enforce restriction on scope of nondeterminism and recursion. Recall that $\beta$ refers to base (non-function) types.}
\label{static-sem-lambdaj}
\end{figure}

\section{Properties}
\label{property}

 \begin{lemma}
     $\mathrm{(Concrete Function)}$. if $\upsilon$ is a value of type $\tau_1\rightarrow\tau_2$, then $\upsilon= \lambda x:\tau_1.e$, where e has type $\tau_2$.
    \end{lemma}
 \begin{proof}
     According to the $\lambdaJ$ syntax, we can get Lemma 1 immediately.
    \end{proof}
 
 \begin{theorem}
     $\mathrm{(Progress).}$ Suppose $e$ is a closed, well-typed expression. Then $e$ is either a value $\upsilon$ or there is some $e'$ such that $\vdash\left\langle\Sigma, \Delta, e\right\rangle\rightarrow\left\langle\Sigma',\Delta',e'\right\rangle$.
 \end{theorem}   
    \begin{proof}
        According to the dynamic demantics \ref{dynamic-sem-lambdaj} and static semantics \ref{static-sem-lambdaj} of $\lambdaJ$, we will proof that any $\lambdaJ$ program holds the \emph{progress} property by structral induction over the syntax of $\lambdaJ$. According to Figure \ref{lambdaj-syntax}, there are nine kinds of expressions including the value expression $\upsilon$. Next we discuss each of them respectively.
        \begin{itemize}
            \item For the value expression $\upsilon$, according to the definition we can conclude that it holds the property immediately.
            \item For the expression $e_1~(op)~e_2$, according to the hyposthesis, we know that sub-expression $e_1$ and $e_2$ are both closed and well-typed expressions. Thus if $e_1$ is not a value, according to E-OP1, there must be a expression $e'_1$ such that $e_1\rightarrow e'_1$. Therefore the expression $e_1~(op)~e_2$ can be reduced to $e'_1~(op)~e_2$, and the same as if the $e_1$ is already a value but $e_2$ is not according to E-OP2. While if both $e_1$ and $e_2$ are values, we know that $e_1~(op)~e_2$ is a value as well. Let $c = e_1~(op)~e_2$, according to E-OP, $\vdash\left\langle\Sigma, \Delta, e_1~(op)~e_2 \right\rangle\rightarrow\left\langle\Sigma',\Delta',c \right\rangle$. Hence, we can conclude that expression $e_1~(op)~e_2$ holds the \emph{progress} property.
            \item For the expression $\textbf{if}~e_1~\textbf{then}~e_t~\textbf{else}~e_f$, the conditional expression $e_1$ is a either a \textbf{concrete} expression or a \textbf{symbolic} expression. If expression $e_1$ is \textbf{concrete} and not a value, according to E-COND, there must be an expression $e'_1$ such that $\vdash\left\langle\Sigma, \Delta, \textbf{if}~e_1~\textbf{then}~e_t~\textbf{else}~e_f \right\rangle\rightarrow\left\langle\Sigma',\Delta',\textbf{if}~e'_1~\textbf{then}~e_t~\textbf{else}~e_f \right\rangle$.  Similarly, if expression $e_1$ is a \textbf{concrete} value, it must be \textbf{true} or \textbf{false}. According to E-CONDTRUE and E-CONDFALSE, $\textbf{if}~e_1~\textbf{then}~e_t~\textbf{else}~e_f$ can be reduced as well. On the other hand, if the expression $e_1$ is \textbf{symbolic}, according to E-CONDSYMT and E-CONDSYMF, it holds the \emph{progress} property.
            \item For the expression $e_1~e_2$, if the sub-expression $e_1$ or $e_2$ is not a value, similar as the expression $e_1~(op)~e_2$, according to E-APP1 and E-APP2, it can be at least reduce one step and evetually both $e_1$ and $e_2$ are values. Then according to the previous \textbf{Concrete Function Lemma}, $e_1~e_2$ must be the form as $\lambda x:e.\upsilon$, according to E-APPLAMBDA, it holds the property immediately.
            \item For the expression $\textbf{defer}~x:\tau\{e\}~\textbf{default}~\upsilon_d$, if sub-expression $e$ is not a value, by the induction hypothesis, there must be an expression $e' $ such that $\vdash\left\langle\Sigma, \Delta, e\right\rangle\rightarrow\left\langle\Sigma',\Delta',e'\right\rangle$, then by E-DEFERCONSTRAINT, the expression can perform a reduction $\vdash\left\langle\Sigma, \Delta, \textbf{defer}~x: \tau\{e\}~\textbf{default}~\upsilon_d \right\rangle\rightarrow\left\langle\Sigma',\Delta',\textbf{defer}~x:\tau\{e'\}~\textbf{default}~\upsilon_d\right\rangle$. In addition, if sub-expression $e$ is a value, according to static semantics, it must be some concrete value $v_c$, then according to E-DEFER, a fresh variable named $x'$ will be generated and a new defaut condition will be added to the $\Delta$ environment, which indicates that the expression \textbf{defer}  is progressive.
            \item For the expression $\textbf{assert}~e$, similar as expression of $\textbf{defer}~x:\tau\{e\}~\textbf{default}~\upsilon_d$, according to E-ASSERTCONSTRAINT and E-ASSERT, it holds the progress property as well.
            \item For the expression $\textbf{concretize}~e~\textbf{with}~\upsilon_c$, if sub-expression $e$ is not a value, according to the hypothesis, there exists an expression $e'$ such that  $\vdash\left\langle\Sigma, \Delta, e\right\rangle\rightarrow\left\langle\Sigma',\Delta',e'\right\rangle$. According to E-CONCRETIZEEXP, it holds the \textbf{progress} property. On the contrary, if sub-expression $e$ is a value, then a MODEL will be built to model the constraints, and it can be reduced to a concrete value $c$ that satisfies the model or an \textbf{error} will be generated while the model cannot be satisfied according to E-CONCRETIZESAT and E-CONCRETIZEUNSAT. Therefore, we can conclude the expression $\textbf{concretize}~e~\textbf{with}~\upsilon_c$ holds the property as well.
            \item For the expression $\textbf{let}~x:\tau=e_1~\textbf{in}~e_2$ and $\textbf{let rec}~f:\tau=e_1~\textbf{in}~e_2$, expressions $e_1$ and $e_2$ are composed by one ore more above expressions. Thus, it holds the property accordingly.
        \end{itemize}
    \end{proof}
    
    \begin{theorem}
        $\mathrm{(Preservision).}$ If $\Gamma\vdash e:\tau\delta$ and $e\rightarrow e'$, then $\Gamma\vdash e':\tau\delta$. 
    \end{theorem}
    \begin{proof}
        According to the static semantics \ref{static-sem-lambdaj}, we can proof that $\lambdaJ$ program is type-preserved by structral induction over the syntax of $\lambdaJ$. According to T-VAR, T-INT, T-BOOL, T-UNIT, T-CONTEXT, we know the simple expressions are all type-preserved. Next we will discuss the $operation$, $condition$, $\lambda~ abstraction$, $application$, $let$, $assert$, $defer$ and $concretize$ expressions respectively. At the beginning, by the typing rules from Figure \ref{static-sem-lambdaj}, we conclude that all the $\delta$ value is the same for both sides except for the T-CONCRETIZE in which, however, the type is also preserved for it only cast the symbolic type to the corresponding concrete type. Therefore, we will not discuss the $\delta$ type in the following proof. In addition, by the hypothesis, all the sub-expressions are well-typed while proof the specific expression.
        \begin{itemize}
            \item For the operation expression $e_1~(op)~e_2$, assume that expression $e_1$ is with the type $\tau_1$ and expression $e_2$ is with the type $\tau_2$, according to T-OP there must be a type $\tau$ that is the common super type of $e_1$ and $e_2$, such that the expression $e_1~(op)~e_2$ is with the type $\tau$. According to E-OP, E-OP1 and E-OP2, after one step evaluation, the expression $e_1$ will be some $e'_1$ with the type $\tau'_1$, such that $\tau'_1<:\tau_1$, or the expression $e_2$  will be reduced to some $e'_2$ with the type $\tau'_2$ and $\tau'_2<:\tau_2$, assume that the evaluated expression is with the type $\tau'$, according to subtyping, we know $\tau'<:\tau$, which indicates that the type is preserved.
            \item For the conditional expression $\textbf{if}~e_1~\textbf{then}~e_t~\textbf{else}~e_f$, assume expressions $e_t$ and $e_f$ are with the type $\tau_1$ and $\tau_2$ respectively. According to T-CONDC and T-CONDSYM, if the expression $e_1$ is with the symbolic type, $\textbf{if}~e_1~\textbf{then}~e_t~\textbf{else}~e_f$ must be with some type $\tau$ which is the common super type of $\tau_1$ and $\tau_2$. Then according to E-CONDSYMT, after one step evaluation $e_t$ will be reduced to some $e'_t$ with the type $\tau'_1$. According to the hypothesis, $\tau'_1<:\tau_1$. Therefore the type $\tau$ is the common super type of $\tau'_1$ and $\tau_2$ as well. And the E-CONDSYMF is the same. In addition, if the expression $e_1$ is with the concrete type, there will be only one branch being taken, either $\tau_1$ or $\tau_2$, either of which holds the subtyping relation  naturally. Consequently, the conditional expression is type-preserved.
            \item For the $\lambda$ abstraction expression $\lambda x:\tau.e$, according to E-APPLAMBDA, the type of the expression will not change and it is type-preserved naturally.
            \item For the application expression $e_1~e_2$, assume the type of $e_1$ and $e_2$ are $\tau_1\rightarrow\tau_2$ and $\tau'_1$ respectively. In addition, $\tau'_1<:\tau_1$, and according to T-APP, the type of application expression is $\tau_2$. Then according to E-APP1 and E-APP2, the evaluated expression will preserve the expression type due to the hypothesis that the sub-expressions $e_1$ and $e_2$ preserve the corresponding type respectively.
            \item Similarly for the expression $\textbf{let rec}~f:\tau=e_1~\textbf{in}~e_2$, $\textbf{assert}~e$, $\textbf{defer}~x:\tau\{e\}~\textbf{default}~\upsilon_d$ and $\textbf{concretize}~e~\textbf{with}~\upsilon_c$.
        \end{itemize}
        
    \end{proof}

\section{Evaluation}
\label{evaluation}
Here is the example area.

\begin{lstlisting}[language=Haskell]
    data Exp = E_BOOL Bool | E_NAT Int 
                    | E_STR String | E_CONST String  
                    | E_VAR Var   | E_CONTEXT 
                    | E_LAMBDA Var Exp | E_THUNK Exp 
                    | E_OP Op Exp Exp  | E_UOP UOp Exp                    
                    | E_IF Exp Exp Exp | E_APP Exp Exp                   
                    | E_DEFER Var Exp | E_ASSERT Exp Exp                  
                    | E_LET Var Exp Exp                   
                    | E_RECORD [(FieldName,Exp)] 
                    | E_FIELD Exp FieldName
                    deriving (Ord,Eq)
    
    data Op = OP_PLUS | OP_MINUS
                    | OP_LESS | OP_GREATER 
                    | OP_EQ | OP_AND | OP_OR | OP_IMPLY
                    deriving (Ord,Eq)
    
    data UOp = OP_NOT   deriving (Ord,Eq) 
    
    data FieldName = FIELD_NAME String  deriving(Ord,Eq)
    data Var = VAR String deriving (Ord,Eq)
    
\end{lstlisting}

\begin{verbatim}
 ---------
 let name =  
     level a in
     policy a: !(context = alice) then bottom in < "Anonymous" | "Alice" >(a)
 
 let msg = "Author is " + name
 
 print {alice} msg
 print {bob} msg
 ------
\end{verbatim}




\section{Conclusion}
\label{conclusion}

The hardest part for us in this course project is trying to understand how the $\lambdaJ$ works. Since the limited page number, the authors only list the necessary semantics and a small part of property proof, eliminating most detailed explanation. The only resource we can find is a presentation video, a github project with thousands of lines of source codes besides this page. We went through several days' heated discussion before we finally get to understand how $\lambdaJ$ rule out the cases of infinite recursion and what \textbf{rep} is designed for. Another non-trivial task is to run the existing codes successfully and know how it works roughly inside.

The most important thing we have learned from this project, is how to design a new programming language for a specific goal, and what we should deal with as it may introduce extra problems. For example, $\lambdaJ$ syntax leads to infinite recursive calls via symbolic expression, so it provides complicated static semantic to constrain the type checking. Also, since all is based on {\ensuremath{\lambda}}-calculus, it makes us understand the power of this simple language deeper. It's also a good chance for us to look back to the teaching materials in previous courses. All these help us to understand types and programming language better.

\section{Acknowledgment}
\label{acknowledgment}

We would like to express our sincere gratitude to Prof.Hu, Prof.Zhao and Prof.Xiong. You have done a splendid teaching work, making very abstract and tough programming language knowledge look straightforward and easy to us. Also, our acknowledgment goes to our TAs, Di Wu and Zhixuan Yang, who have sacrificed lots of time to help us in learning this course.

\label{reference}

% The bibliography should be embedded for final submission.
%\bibliographystyle{abbrvnat}
\begin{thebibliography}{00}

\bibitem[1]{popl12}
J. Yang, K. Yessenov, and A. Solar-Lezama. 
A language for automatically enforcing privacy policies. In Proceedings of the 39th annual ACM SIGPLAN-SIGACT symposium on Principles of programming languages (POPL '12). ACM, New York, NY, USA, 85-96.
    
\bibitem[2]{tpl}
B. Pierce. Types and Programming Languages, \emph{MIT Press}, 2002.
\end{thebibliography}
 
\end{document}
    